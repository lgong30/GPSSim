% Template for drawing GPS figure
% Author: Long Gong
\documentclass[border=2pt]{standalone}
%%%<
\usepackage{verbatim}
%%%>
\begin{comment}
:Title: Template for GPS Figure
:Author: Long Gong

A template for GPS figure. 

In some ways, this TeX script works as the "model" of our application 
for visualizing a GPS simulation results. 

Parameters:
    width: scalar, total length of x axis
    height: scalar, total length of y axis
    xlabels: list, labels for x axis
    ylabels: list, labels for y axis
    flows: list of dict, packets in each flow, its detailed structures 
           look like as follows,
            [{
               "id": <flow id>,
               "packets": [
               {
                "v_start_time": <virtual start time>,
                "v_finish_time": <virtual finish time>,
                "y_min": <bottom border value>,
                "y_max": <top border value>
               },
               ...
               ] 
            },
            ...
            ]

Programmed in TikZ by Long Gong. Templating language is Jinja2, 
templaing syntax is the default setting of Jinja2.
\end{comment}

\usepackage{tikz}
\usetikzlibrary{shapes, positioning}



\begin{comment}
%% detailed results %%
{{ detailed_results }}
\end{comment}

\begin{document}
\begin{tikzpicture}


%% place x axis label

    \draw ({{ xl }}, 1pt) -- ({{ xl }}, -1pt) node[anchor=north] {{"{"}}${{ xl }}${{"}"}};


%% place y axis label

    \draw (1pt, {{ yl }}) -- (-1pt, {{ yl }}) node[anchor=east] {{"{"}}flow ${{ yl }}${{"}"}};


% flows

% flow {{ flow.id }}

\filldraw[fill=white, draw=black] ({{ pkt.v_start_time }}, {{ pkt.y_min }} ) rectangle ({{ pkt.v_finish_time }}, {{ pkt.y_max }});



%% virtual time axis (remove labels)
% node[anchor=north west] {}
%  node[anchor=south east] {}
\draw[thick,->] (0,0) -- ({{ width }},0) ;
\draw[thick,->] (0,0) -- (0,{{ height }});

\end{tikzpicture}
\end{document}